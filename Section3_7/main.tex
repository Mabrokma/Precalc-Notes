\documentclass{tufte-handout}

\title{3.7 Rational functions}

\author[AW]{Ammon Washburn}

\usepackage{graphicx} % allow embedded images
  \setkeys{Gin}{width=\linewidth,totalheight=\textheight,keepaspectratio}
  \graphicspath{{graphics/}} % set of paths to search for images
\usepackage{amsmath}  % extended mathematics
\usepackage{booktabs} % book-quality tables
\usepackage{units}    % non-stacked fractions and better unit spacing
\usepackage{multicol} % multiple column layout facilities
\usepackage{lipsum}   % filler text
\usepackage{enumerate}
\usepackage{wrapfig}
\usepackage{fancyvrb} % extended verbatim environments
  \fvset{fontsize=\normalsize}% default font size for fancy-verbatim environments
  \usepackage{tikz}

% Standardize command font styles and environments
\newcommand{\doccmd}[1]{\texttt{\textbackslash#1}}% command name -- adds backslash automatically
\newcommand{\docopt}[1]{\ensuremath{\langle}\textrm{\textit{#1}}\ensuremath{\rangle}}% optional command argument
\newcommand{\docarg}[1]{\textrm{\textit{#1}}}% (required) command argument
\newcommand{\docenv}[1]{\textsf{#1}}% environment name
\newcommand{\docpkg}[1]{\texttt{#1}}% package name
\newcommand{\doccls}[1]{\texttt{#1}}% document class name
\newcommand{\docclsopt}[1]{\texttt{#1}}% document class option name
\newenvironment{docspec}{\begin{quote}\noindent}{\end{quote}}% command specification environment

\newtheorem{mydef}{Definition}
\providecommand{\floor}[1]{\left \lfloor #1 \right \rfloor }

\begin{document}
\maketitle


\begin{abstract}
Learn some key properties about rational functions and learn how to graph them
\end{abstract}

\section{Properties of Rational Functions}

We are going to learn some key properties about rational functions

\begin{itemize}
\item Domain and Vertical Asymptotes
\item Horizontal Asymptotes
\item Zeros and y-intercepts
\item Slant Asymptotes
\end{itemize}

\subsection{Domain and Vertical Asymptotes}

If you know where a function is undefined then you know the vertical asymptotes.  We write asymptotes as vertical lines $x=c$.  Factor out the bottom.

If $r(c) = \frac{0}{0}$ then it is called a removable singularity.  It means you could almost remove it (but don't!).

\subsection{Horizontal Asymptotes}

We can find the horizontal asymptotes by using long division.  If it happens that we have two cases then we don't have to do long division.

Case 1: If the degree of the top is smaller than the degree of the bottom. $y=0$ is the horizontal asymptote.  

Case 2: If the degree of the top is equal to the degree of the bottom then the horizontal asymptote is the ratio of the coefficients.

Case 3: If the degree of the top is greater than the degree of the bottom then we need to do long division.  Then we can get slant asymptote or something greater. $y=ax+b$.

\section{Graphing Rational Functions and Examples}
Steps to graph rational functions
\begin{itemize}
\item Find domain and vertical asymptotes
\item Find zeros and y-intercept
\item Find the horizontal asymptote
\item Use as many points as needed to figure out where graph is
\end{itemize}

Examples
\begin{multicols}{2}
\begin{enumerate}[(a)]
\item $r(x) = \frac{2x^2+7x-4}{x^2+x-2}$
\item $t(x) = \frac{2x^3+2x}{x^2-1}$
\item $s(x) = \frac{x^2-4}{x^2-x-6}$
\item $a(x) = \frac{x-2}{x^2-16}$
\item $h(x) = \frac{3x^2+9x-30}{4x^2+28x+10}$
\end{enumerate}
\end{multicols}

\end{document}