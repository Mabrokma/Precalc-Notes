\documentclass{tufte-handout}

\title{Section 2.5 Transformations}

\author[AW]{Ammon Washburn}

\usepackage{graphicx} % allow embedded images
  \setkeys{Gin}{width=\linewidth,totalheight=\textheight,keepaspectratio}
  \graphicspath{{graphics/}} % set of paths to search for images
\usepackage{amsmath}  % extended mathematics
\usepackage{booktabs} % book-quality tables
\usepackage{units}    % non-stacked fractions and better unit spacing
\usepackage{multicol} % multiple column layout facilities
\usepackage{lipsum}   % filler text
\usepackage{fancyvrb} % extended verbatim environments
  \fvset{fontsize=\normalsize}% default font size for fancy-verbatim environments

% Standardize command font styles and environments
\newcommand{\doccmd}[1]{\texttt{\textbackslash#1}}% command name -- adds backslash automatically
\newcommand{\docopt}[1]{\ensuremath{\langle}\textrm{\textit{#1}}\ensuremath{\rangle}}% optional command argument
\newcommand{\docarg}[1]{\textrm{\textit{#1}}}% (required) command argument
\newcommand{\docenv}[1]{\textsf{#1}}% environment name
\newcommand{\docpkg}[1]{\texttt{#1}}% package name
\newcommand{\doccls}[1]{\texttt{#1}}% document class name
\newcommand{\docclsopt}[1]{\texttt{#1}}% document class option name
\newenvironment{docspec}{\begin{quote}\noindent}{\end{quote}}% command specification environment

\newtheorem{mydef}{Definition}
\providecommand{\floor}[1]{\left \lfloor #1 \right \rfloor }

\begin{document}
\maketitle

\begin{abstract}
Learn how to do transformations with functions
\end{abstract}

\section{Definitions of Transformations}
For the following examples assume we have a function w(x).  We only know one point (3,2)
\begin{mydef}
A vertical transformation of f(x) by c is f(x) + c
\end{mydef}
If $c<0$ then we call this shifting f(x) down by $|c|$ but if $c>0$ then we call it shifting f(x) up by c.

Example: If we have w(x) + 2 then the point moves to (3,4)
\begin{mydef}
A horizontal transformation of f(x) by c is f(x+c)
\end{mydef}

If $c<0$ then the function is shifted to the \textit{right}.  If $c>0$ then the function is shifted to the \textit{left}.

Example: If we have w(x+2) then the point moves to (1,2)

\begin{mydef}
Reflection: -f(x) reflects the graph across the x-axis and f(-x) reflects the graph across the y-axis.
\end{mydef}

Example: If we have -w(x) then the point moves to (3,-2).  If we have w(-x) then the point moves to (-3,2).

\begin{mydef}
A horizontal stretch/compression of f by c is cf(x). If c<1 then it is a compression and if c>1 then it is a stretching
\end{mydef}

Example: If we have .5w(x) then the point moves to (3,1).  This is a compression.  If we have 2w(x) then the point moves to (3,4).  This is a stretching.

\begin{mydef}
A vertical stretch/compression of f by c is f(cx). If c<1 then it is a stretch and if c>1 then it is a compression
\end{mydef}

Example: If we have w(.5x) then the point moves to (6,2).  This is a stretching.  If we have w(2x) then the point moves to (1.5,2). This is a compression.

\section{Combining Transformations}
When you several different transformations, it is imperative that you do them in the right order or you will get a different answer.
\subsection{Transformations on y}
Follow natural order of operations.  Consider -2(w(x) +1).  $(3,4) \rightarrow (3,5) \rightarrow (3,10) \rightarrow (3,-10)$.  Note you only affect the y-coordinate.
\subsection{Transformation on x}
Follow the reverse order of operations.  It is kind of confusing.  Consider w(-2(x+1)). $(3,4) \rightarrow (-3,4) \rightarrow (-1.5,4) \rightarrow (-2.5,4)$
\subsection{Transformations on both x and y}
You must do the transformations in x in the right order and you must do transformations in y in the right order.  However, if you do the x transformation first or the y transformations first it doesn't matter.

\section{Even and Odd functions}
\begin{mydef}
An even function f(x) is a function where f(-x) = f(x).  In other words the function has a symmetry about the y-axis
\end{mydef}
\begin{mydef}
An odd function f(x) is a function where f(-x) = -f(x). In other words, if you rotate the function about the origin 180 degrees then you get the same function.
\end{mydef}

\subsection{Examples}
$f(x) = x^n$ is odd if n is odd and even if n is even.  Let's show this.

Let n be even.  $f(-x) = (-x)^n = (-1)^n x^n = x^n = f(x)$.

Let n be odd. $f(-x) = (-x)^n = (-1)^n x^n = -x^n = -f(x)$.

Problems:
\begin{eqnarray}
a) \hspace{2 mm} d(x) = 3x^3 - x \hspace{5mm} b) \hspace{2 mm} a(x) = x^3 - 1 \hspace{5mm} c) \hspace{2mm} l(x) =-x^4 + 5x^2 +3 \\
answers: \hspace{2mm} a) \hspace{2mm} Odd \hspace{5mm} b) \hspace{2mm} Neither \hspace{5mm} c) \hspace{2mm} Even
\end{eqnarray}














\end{document}