\documentclass{tufte-handout}

\title{3.3 Dividing polynomials}

\author[AW]{Ammon Washburn}

\usepackage{graphicx} % allow embedded images
  \setkeys{Gin}{width=\linewidth,totalheight=\textheight,keepaspectratio}
  \graphicspath{{graphics/}} % set of paths to search for images
\usepackage{amsmath}  % extended mathematics
\usepackage{booktabs} % book-quality tables
\usepackage{units}    % non-stacked fractions and better unit spacing
\usepackage{multicol} % multiple column layout facilities
\usepackage{lipsum}   % filler text
\usepackage[inline]{enumitem}
\usepackage{wrapfig}
\usepackage{fancyvrb} % extended verbatim environments
  \fvset{fontsize=\normalsize}% default font size for fancy-verbatim environments
  \usepackage{tikz}

% Standardize command font styles and environments
\newcommand{\doccmd}[1]{\texttt{\textbackslash#1}}% command name -- adds backslash automatically
\newcommand{\docopt}[1]{\ensuremath{\langle}\textrm{\textit{#1}}\ensuremath{\rangle}}% optional command argument
\newcommand{\docarg}[1]{\textrm{\textit{#1}}}% (required) command argument
\newcommand{\docenv}[1]{\textsf{#1}}% environment name
\newcommand{\docpkg}[1]{\texttt{#1}}% package name
\newcommand{\doccls}[1]{\texttt{#1}}% document class name
\newcommand{\docclsopt}[1]{\texttt{#1}}% document class option name
\newenvironment{docspec}{\begin{quote}\noindent}{\end{quote}}% command specification environment

\newtheorem{mydef}{Definition}
\providecommand{\floor}[1]{\left \lfloor #1 \right \rfloor }

\begin{document}
\maketitle

\begin{abstract}
Learn how to use long and synthetic division to split up polynomials.  Learn how to factor large polynomials.
\end{abstract}

\section{Division Algorithm}

If you divide 7 by 2, then you get 3 with a remainder of 1.  In other words, how can you write 7 in terms of the largest multiple of 2?  $7 = 3 * 2 + 1$.  Simple right?  What about polynomials?


\begin{mydef}
Division Algorithm: For two polynomials $P(x)$ and $D(x)$ where you divide $P$ (dividend) by $D$ (divisor).  There are $Q(x)$ (quotient) and $R(x)$ (remainder) such that $P(x) = Q(x)D(x) + R(X)$ and the degree of $R$ is less than the degree of $D$.
\end{mydef}

Alternatively we could write it as $\frac{P(x)}{D(x)}= Q(x) + \frac{R(x)}{D(x)}$

Algorithm? What algorithm?  You can use long division or synthetic division.

\subsection{Examples}

Long division: Divide $7x^4-3x^2+2x$ by $x^2+x$

Answer: $Q(x) = 7x^2-7x+4$ and $R(x) = -2x$.  How can we write the dividend in terms of the divisor?

Long division: Divide $x^5-2x^3+2x^2+3x-4$ by $x^3+2$.  Then by $x^2-2$.  Answer: The other quotient with remainder $3x$

\section{Synthetic Division}

This is only used when the divisor is of the form $x-c$.  

Let's take $7x^4-3x^2+2x$ and divide by $x - 0$.  Answer: $7x^3-3x+2$ with remainder of $0$.

Now take $7x^3-3x+2$ and divide by $x-(-1)$.  Answer: $7x^2-7x+4$ with remainder of $2$.

\subsection{Remainder Theorem}

\begin{mydef}
Remainder Theorem: When you divide a polynomial $P(x)$ by a linear divisor $x-c$ then the remainder is $P(c)$.
\end{mydef}

Check above examples to prove.  Note that with $x-0$ we get a special case.

\begin{mydef}
Factor Theorem: $x-c$ is a factor of $P(x)$ if and only if $P(c) = 0$.
\end{mydef}

Using factor theorem: Find $h$ so that $x-2$ is a factor of $P(x) = 2x^4 - 9x^3 - h x^2 + 36x-16$

\section{Examples}
Factor the following polynomials with the given divisors and factor out completely from there.

$2 x^4-5 x^3-23 x^2+38 x+24$ with $x+3$ and $x-4$

$3 x^5-8 x^4+15 x^3-30 x^2+12 x+8$ with $x^2+4$ and find another factor using your calculator and then check it.

\end{document}