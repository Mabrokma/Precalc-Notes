\documentclass{tufte-handout}

\title{Section 7.1 Trig Identities}

\author[AW]{Ammon Washburn}

\usepackage{graphicx} % allow embedded images
  \setkeys{Gin}{width=\linewidth,totalheight=\textheight,keepaspectratio}
  \graphicspath{{graphics/}} % set of paths to search for images
\usepackage{amsmath}  % extended mathematics
\usepackage{booktabs} % book-quality tables
\usepackage{units}    % non-stacked fractions and better unit spacing
\usepackage{multicol} % multiple column layout facilities
\usepackage{lipsum}   % filler text
\usepackage{enumerate}
\usepackage{wrapfig}
\usepackage{fancyvrb} % extended verbatim environments
  \fvset{fontsize=\normalsize}% default font size for fancy-verbatim environments
\usepackage{tikz}
\usepackage{subcaption}
\captionsetup{compatibility=false}
\usepackage{mathtools}
\usepackage{graphicx}
\usepackage{amssymb}
\usepackage{enumerate}
\usepackage{color}
\usepackage{fancyvrb}
\usepackage{breqn}
\usepackage{fancyhdr}
\usepackage{multicol}
%\usepackage[latin1]{inputenc}
\usepackage{tikz}
\usepackage{pgfplots}
\pgfplotsset{compat=1.8}

\definecolor{dkgreen}{rgb}{0,0.6,0}
\definecolor{gray}{rgb}{0.5,0.5,0.5}
\definecolor{mauve}{rgb}{0.58,0,0.82}

\newcommand{\R}[1]{\mathbb{R}^{#1}}

\pgfplotsset{vasymptote/.style={
    before end axis/.append code={
        \draw[densely dashed] ({rel axis cs:0,0} -| {axis cs:#1,0})
        -- ({rel axis cs:0,1} -| {axis cs:#1,0});
    }
}}
\pgfplotsset{hasymptote/.style={
    before end axis/.append code={
    	%\draw (axis cs:0,1) -- ({axis cs:0,1}-|{rel axis cs:1,0});
        \draw[densely dashed] ({rel axis cs:0,1} -| {axis cs:0,#1})
        -- ({rel axis cs:0,0} -| {axis cs:0,#1});
    }
}}

% Standardize command font styles and environments
\newcommand{\doccmd}[1]{\texttt{\textbackslash#1}}% command name -- adds backslash automatically
\newcommand{\docopt}[1]{\ensuremath{\langle}\textrm{\textit{#1}}\ensuremath{\rangle}}% optional command argument
\newcommand{\docarg}[1]{\textrm{\textit{#1}}}% (required) command argument
\newcommand{\docenv}[1]{\textsf{#1}}% environment name
\newcommand{\docpkg}[1]{\texttt{#1}}% package name
\newcommand{\doccls}[1]{\texttt{#1}}% document class name
\newcommand{\docclsopt}[1]{\texttt{#1}}% document class option name
\newenvironment{docspec}{\begin{quote}\noindent}{\end{quote}}% command specification environment
\newcommand{\Z}[1]{\mathbb{Z}^{#1}}

\newtheorem{mydef}{Definition}
\providecommand{\floor}[1]{\left \lfloor #1 \right \rfloor }
\providecommand{\abs}[1]{| #1 |}


\begin{document}

\maketitle

\begin{abstract}
Learning more trig identities
\end{abstract}
These Trig identities will not be given on the final.

Old Trig Identities:
\begin{align*}
\csc(x) & = \frac{1}{\sin(x)} & \sec(x) & = \frac{1}{\cos(x)} & \cot(x) & = \frac{1}{\tan(x)} \\
\tan(x) & = \frac{\sin(x)}{\cos(x)} & \cot(x) & = \frac{\cos(x)}{\sin(x)} \\
\sin^2(x) + \cos^2(x) & = 1 & \tan^2(x) + 1 & = \sec^2(X) & 1 + \cot^2(x) & = \csc^2(x) \\
\sin(-x) & = -\sin(x) & \cos(-x) &= \cos(x) & \tan(-x) & = - \tan(x)
\end{align*}


Cofunction Identities:
\begin{align*}
\sin(\frac{\pi}{2}-u) & = \cos(u) & \tan(\frac{\pi}{2}-u) & = \cot(u) & \sec(\frac{\pi}{2}-u) & = \csc(u) \\
\cos(\frac{\pi}{2}-u) & = \sin(u) & \cot(\frac{\pi}{2}-u) & = \tan(u) & \csc(\frac{\pi}{2}-u) & = \sec(u) 
\end{align*}

\section{Simplifying Trigonometric Expressions}
Simplify $\frac{\sin(\theta)}{\cos(\theta)} + \frac{\cos(\theta)}{1+\sin(\theta)}$.

\section{Proving Trig Identities}
Identity versus Equation.  An identity is true no matter what $x$ you put in.  An equation you have to figure which $x$ that makes it true.  In order words an identity is an equation that is true no matter what $x$ you put in.

Guidelines for proving Trig identities:
\begin{enumerate}
\item Start with one side.  Try to make it look like the other side. If stuck try the other side.
\item Use known identities.  Use them to change both sides to look like the other one.
\item Convert to sines and cosines. This is helpful if stuck.
\end{enumerate}

Since we are proving identities we do not want to change the original thing we are proving.

\subsection{Examples}
\begin{enumerate}[(a)]
\item $2\tan(x)\sec(x) = \frac{1}{1-\sin(x)}- \frac{1}{1+\sin(x)}$
\item $\frac{\cos(u)}{1-\sin(u)} = \sec(u) + \tan(u)$
\item Work both sides for this one $\frac{1 + \cos(\theta)}{\cos(\theta)} = \frac{\tan^2(\theta)}{\sec(\theta)-1}$
\end{enumerate}

\section{Trig Substitution}
This will be useful in Calculus

\begin{enumerate}[(a)]
\item $\sqrt[]{x^2 -1}$ where $x = \sec(\theta)$ and $0 \leq \theta \leq \frac{\pi}{2}$
\item $\frac{1}{x^2\sqrt[]{4+x^2}}$ where $x = 2\tan(\theta)$ and $0 \leq \theta \leq \frac{\pi}{2}$
\end{enumerate}

\end{document}