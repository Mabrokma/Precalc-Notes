\documentclass{tufte-handout}

\title{4.5 Exponential and Logarithmic equations}

\author[AW]{Ammon Washburn}

\usepackage{graphicx} % allow embedded images
  \setkeys{Gin}{width=\linewidth,totalheight=\textheight,keepaspectratio}
  \graphicspath{{graphics/}} % set of paths to search for images
\usepackage{amsmath}  % extended mathematics
\usepackage{booktabs} % book-quality tables
\usepackage{units}    % non-stacked fractions and better unit spacing
\usepackage{multicol} % multiple column layout facilities
\usepackage{lipsum}   % filler text
\usepackage{enumerate}
\usepackage{wrapfig}
\usepackage{fancyvrb} % extended verbatim environments
  \fvset{fontsize=\normalsize}% default font size for fancy-verbatim environments
  \usepackage{tikz}

% Standardize command font styles and environments
\newcommand{\doccmd}[1]{\texttt{\textbackslash#1}}% command name -- adds backslash automatically
\newcommand{\docopt}[1]{\ensuremath{\langle}\textrm{\textit{#1}}\ensuremath{\rangle}}% optional command argument
\newcommand{\docarg}[1]{\textrm{\textit{#1}}}% (required) command argument
\newcommand{\docenv}[1]{\textsf{#1}}% environment name
\newcommand{\docpkg}[1]{\texttt{#1}}% package name
\newcommand{\doccls}[1]{\texttt{#1}}% document class name
\newcommand{\docclsopt}[1]{\texttt{#1}}% document class option name
\newenvironment{docspec}{\begin{quote}\noindent}{\end{quote}}% command specification environment

\newtheorem{mydef}{Definition}
\providecommand{\floor}[1]{\left \lfloor #1 \right \rfloor }

\begin{document}
\maketitle

\begin{abstract}
We will learn how to solve logarithmic and exponential equations
\end{abstract}

\section{Section 4.5: Exponential and Logarithmic Equations}

\subsection{Exponential Equations (variable in exponent)}
Guidelines for solving exponential equations:
\begin{enumerate}
\item Isolate the exponential expression on one side of equation
\item Take the log of each side
\item Solve for the variable
\end{enumerate}

\subsection{Examples}
\begin{enumerate}
\item Solve for $x$: $5^{3 - 7x} = 24\cdot 5^{1-2x}$ (round to 6 decimal places).
\begin{align*}
24 &= 5^{3-7x}5^{2x-1} = 5^{2-5x} \Rightarrow 
\log(24) = (2-5x)\log(5) \Rightarrow
2-5x = \frac{\log(24)}{\log(5)} \Rightarrow \\
5x &= 2 - \frac{\log(24)}{\log(5)} \Rightarrow 
x = \frac{2}{5} - \frac{\log(24)}{5\log(5)} 
\approx 0.005073
\end{align*}
Check: $5^{3 - 7(0.005073)} \approx 118.056$ and $24\cdot 5^{1-2(0.005073)} \approx 118.056$.

\item $3e^x = 8 \Rightarrow e^x = \frac{8}{3} \Rightarrow x = \ln(8/3) \approx 0.9808$.
Check via graphing $y_1 = 3e^x$ and $y_2 = 8$. (use option ``intersect" in TRACE$>$CALC menu). Can be done in general ($y_1 = LHS, y_2 = RHS$)

\item $x^2 2^x - 2^x = 0$.
\begin{align*}
0 &= (x^2 - 1)2^x
\end{align*}
Since $2^x > 0$ for all $x$, then $(x^2 - 1) = 0 \Rightarrow x^2 = 1 \Rightarrow x = \pm 1$ (check both answers).
\end{enumerate}

\subsection{Logarithmic Equations (variable in logarithm)}
Guidelines for solving logarithmic equations:
\begin{enumerate}
\item Isolate the log term on one side of equation (only one log term is best)
\item Plug both sides into a log (ln or log)
\item Solve for the variable
\item Check your answer in the original equation!
\end{enumerate}

\subsection{Examples}
\begin{enumerate}
\item $\log(x + 2) + \log(x-1) = 1$
\begin{align*}
\log((x+2)(x-1)) = 1 \Rightarrow 
(x+2)(x-1) = 10 \Rightarrow
x^2 + x - 2 = 10 \Rightarrow \\
x^2 + x - 12 = 0 \Rightarrow
(x-3)(x+4) = 0 \Rightarrow
x = 3, -4
\end{align*}
check both answers! only $x = 3$ works in original equation.

\item $\log_2(x) + \log_4(x-2) = 1$  Use change of base formula to get bases of logs the same and then combine.

\item If $I_0$ and $I$ denote the intensity of light before and after going through a material and $x$ is the distance (in feet) the light travels in the material, then according to the Beer-Lambert Law, 
\begin{align*}
-\frac{1}{k} \ln\left( \frac{I}{I_0} \right) = x
\end{align*}
where $k$ is a constant depending on the type of material.
\begin{enumerate}[(a)]
\item Solve for $I$.
\item For a certain lake $k = 0.025$, and $I_0 = 14$ lumens (lm). Find the light intensity at a depth of 20 feet.
\end{enumerate}
\begin{enumerate}[(a)]
\item \begin{align*}
\ln\left( \frac{I}{I_0} \right) = -kx \Rightarrow 
\frac{I}{I_0} = e^{-kx} \Rightarrow
I = I_0 e^{-kx}
\end{align*}

\item Using our formula,
\begin{align*}
I = 14 \cdot e^{-0.025\cdot 20} \approx 8.49
\end{align*}
what are the units?
\end{enumerate}
\end{enumerate}

\subsection{Compound Interest}

\subsection{Example}
Doubling time of an investment.
You invest \$9000 in an account with an interest rate of 5\% per year.
How many years will it take for the investment to double if the interest is compounded (a) quarterly, or (b) continously?
\begin{enumerate}[(a)]
\item \begin{align*}
18000 = 9000 \left( 1 + \frac{0.05}{4} \right)^{4\cdot t} \Rightarrow
2 = (1.0125)^{4t} \Rightarrow
4t = \log(2)/\log(1.0125) \Rightarrow \\
t = \frac{\log(2)}{4\log(1.0125)}
\approx 13.95 \; yrs
\end{align*}
\item  \begin{align*}
18000 = 9000 e^{0.05t} \Rightarrow
2 = e^{0.05t} \Rightarrow
0.05t = \ln(2) \Rightarrow
t = 20\ln(2) \approx 13.86 \; yrs
\end{align*}
\end{enumerate}
\end{document}