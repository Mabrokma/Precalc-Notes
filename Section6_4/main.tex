\documentclass{tufte-handout}

\title{Section 6.4 Inverse Trigonometric functions of angles}

\author[AW]{Ammon Washburn}

\usepackage{graphicx} % allow embedded images
  \setkeys{Gin}{width=\linewidth,totalheight=\textheight,keepaspectratio}
  \graphicspath{{graphics/}} % set of paths to search for images
\usepackage{amsmath}  % extended mathematics
\usepackage{booktabs} % book-quality tables
\usepackage{units}    % non-stacked fractions and better unit spacing
\usepackage{multicol} % multiple column layout facilities
\usepackage{lipsum}   % filler text
\usepackage{enumerate}
\usepackage{wrapfig}
\usepackage{fancyvrb} % extended verbatim environments
  \fvset{fontsize=\normalsize}% default font size for fancy-verbatim environments
\usepackage{tikz}
\usepackage{subcaption}
\captionsetup{compatibility=false}
\usepackage{mathtools}
\usepackage{graphicx}
\usepackage{amssymb}
\usepackage{enumerate}
\usepackage{color}
\usepackage{fancyvrb}
\usepackage{breqn}
\usepackage{fancyhdr}
\usepackage{multicol}
%\usepackage[latin1]{inputenc}
\usepackage{tikz}
\usepackage{pgfplots}
\pgfplotsset{compat=1.8}

\definecolor{dkgreen}{rgb}{0,0.6,0}
\definecolor{gray}{rgb}{0.5,0.5,0.5}
\definecolor{mauve}{rgb}{0.58,0,0.82}

\newcommand{\R}[1]{\mathbb{R}^{#1}}

\pgfplotsset{vasymptote/.style={
    before end axis/.append code={
        \draw[densely dashed] ({rel axis cs:0,0} -| {axis cs:#1,0})
        -- ({rel axis cs:0,1} -| {axis cs:#1,0});
    }
}}
\pgfplotsset{hasymptote/.style={
    before end axis/.append code={
    	%\draw (axis cs:0,1) -- ({axis cs:0,1}-|{rel axis cs:1,0});
        \draw[densely dashed] ({rel axis cs:0,1} -| {axis cs:0,#1})
        -- ({rel axis cs:0,0} -| {axis cs:0,#1});
    }
}}

% Standardize command font styles and environments
\newcommand{\doccmd}[1]{\texttt{\textbackslash#1}}% command name -- adds backslash automatically
\newcommand{\docopt}[1]{\ensuremath{\langle}\textrm{\textit{#1}}\ensuremath{\rangle}}% optional command argument
\newcommand{\docarg}[1]{\textrm{\textit{#1}}}% (required) command argument
\newcommand{\docenv}[1]{\textsf{#1}}% environment name
\newcommand{\docpkg}[1]{\texttt{#1}}% package name
\newcommand{\doccls}[1]{\texttt{#1}}% document class name
\newcommand{\docclsopt}[1]{\texttt{#1}}% document class option name
\newenvironment{docspec}{\begin{quote}\noindent}{\end{quote}}% command specification environment
\newcommand{\Z}[1]{\mathbb{Z}^{#1}}

\newtheorem{mydef}{Definition}
\providecommand{\floor}[1]{\left \lfloor #1 \right \rfloor }
\providecommand{\abs}[1]{| #1 |}


\begin{document}

\maketitle

\begin{abstract}
Getting more familiar with inverse trig functions using right triangles
\end{abstract}

\section{Inverse Sine, Inverse Cosine, and Inverse Tangent}

If we are given a point and the right triangle associated with that point with hypotenuse $r$ then we can define inverse trig functions in the following way.

\begin{align*}
\sin^{-1}(\frac{y}{r}) &= \theta & Domain:& [-1,1] & Range:&[-\frac{\pi}{2},\frac{\pi}{2}] \\
\cos^{-1}(\frac{x}{r}) &= \theta & Domain:&[-1,1] & Range:&[0,\pi] \\
\tan^{-1}(\frac{y}{x}) &= \theta & Domain:&(-\infty,\infty) & Range:&(-\frac{\pi}{2},\frac{\pi}{2})
\end{align*}

\section{Solving for Angles in Right Triangles}

Given two bits of information (Either $a,b$, $b,r$, $a,r$) You should be able to use the inverse trig functions to find the angle.

\subsection{Examples}
\begin{enumerate}
\item A 40ft ladder leans against a building and the base is ft from the building.  What is the angle between the ladder and the building?
\item Find all the angles $\theta$ between $0$ and $180$ degrees satisfying
\begin{enumerate}
\item $\sin(\theta) = 0.4$
\item $\cos(\theta) = 0.4$
\end{enumerate}
\end{enumerate}

\section{Evaluating Expressions Involving Inverse Trigonometric Functions}

Steps to solve a nested Inverse Trig composition
\begin{itemize}
\item Draw the triangle and label $\theta$ appropriately.
\item Then evaluate the outer functions based on $\theta$.
\end{itemize}

\subsection{Examples}

\begin{enumerate}[(a)]
\item $\cos(\sin^{-1}(\frac{3}{5}))$
\item $\tan(\cos^{-1}(x))$ for $-1 \leq x \leq 1$
\item Problems 39 or 40 in the textbook
\end{enumerate}

\end{document}