\documentclass{tufte-handout}

\title{Section 2.1 Functions}

\author[AW]{Ammon Washburn}

\usepackage{graphicx} % allow embedded images
  \setkeys{Gin}{width=\linewidth,totalheight=\textheight,keepaspectratio}
  \graphicspath{{graphics/}} % set of paths to search for images
\usepackage{amsmath}  % extended mathematics
\usepackage{booktabs} % book-quality tables
\usepackage{units}    % non-stacked fractions and better unit spacing
\usepackage{multicol} % multiple column layout facilities
\usepackage{lipsum}   % filler text
\usepackage{fancyvrb} % extended verbatim environments
  \fvset{fontsize=\normalsize}% default font size for fancy-verbatim environments

% Standardize command font styles and environments
\newcommand{\doccmd}[1]{\texttt{\textbackslash#1}}% command name -- adds backslash automatically
\newcommand{\docopt}[1]{\ensuremath{\langle}\textrm{\textit{#1}}\ensuremath{\rangle}}% optional command argument
\newcommand{\docarg}[1]{\textrm{\textit{#1}}}% (required) command argument
\newcommand{\docenv}[1]{\textsf{#1}}% environment name
\newcommand{\docpkg}[1]{\texttt{#1}}% package name
\newcommand{\doccls}[1]{\texttt{#1}}% document class name
\newcommand{\docclsopt}[1]{\texttt{#1}}% document class option name
\newenvironment{docspec}{\begin{quote}\noindent}{\end{quote}}% command specification environment

\newtheorem{mydef}{Definition}

\begin{document}
\maketitle

\begin{abstract}
In this section we will try to understand what functions are and how to find various properties of functions.  Functions are what we use to model the real world. 
\end{abstract}
\begin{marginfigure}
  \includegraphics[width=\linewidth]{2-1FunctionDef.png}
  \caption{Functions can be thought of as a rule between sets or a machine.  There can't be two outputs associated with function.}
  \label{fig:functionsets}
\end{marginfigure}
\section{ Properties of functions}
\subsection{What is a function?}
\begin{mydef}
Function is a rule which given an input it only has one output associated with it.
\end{mydef} 
Beware of tables in which the same input appears twice but with different outputs.  Also if there is a + and - then it is not a function.  The inputs are also called domain and preimage.  The outputs are also called range and image.
\begin{mydef}
The independent variable is a symbol representing an arbitrary element of the domain.  The dependent variable is a symbol representing an arbitrary element of the range.\sidenote[]{The terms independent and dependent come from the fact that in a function, the output that comes out of the function depends on the input. }
\end{mydef}

\subsection{Piece-wise defined function}
Consider
\[
f(x) = \begin{cases}
x+2 & x \leq -2 \\
x^2 & x > -2 \\

\end{cases}
\]

This function is defined in pieces.  There is a normal function on the left and then where that function has hold in on the right. \begin{marginfigure}
  \includegraphics[width=\linewidth]{2-1Piece-wise.png}
  \caption{Graph of a piece-wise defined function.}
  \label{fig:functionsets}
\end{marginfigure}
It is easy to find the domain from a piece-wise function because they tell you.  You just have to add up all the pieces together.
\subsection{Finding the domain of a function}
The domain of the function is all the inputs that make sense to put in the function.  Examples of bad inputs.
Consider the not so nice function
\[f(x) = \frac{\sqrt[]{x+5}}{x-2}\]
What are some numbers that it wouldn't make sense to plug in for $x$?
Some bad values are 2, -6, -10.  Some good values are -5, 1.9999, 1.

To know whether values are bad you will get familiar with certain types of functions and the problems they have. Rational, radical, and logarithmic functions all have certain values that are bad and you will get to know them well.

It is also possible that they will explicitly tell you the allowed values of the domain for example $h(x) = x^2$ for $ 0 \leq x < \infty$
\subsection{Examples}
\begin{equation}\label{ex:walk}
h(r) = \frac{r}{\sqrt[]{8-2r-r^2}}
\end{equation}

We want $8-2r-r^2 \geq 0$ because of the square root.

We change it to $8-2r-r^2 > 0 $ so that we don't divide by zero.

After factoring we have $(r+4)(2-r) > 0$.  Don't do $r>4$ and $r<2$ because that is wrong!  Instead check three points $r=-5,0,3$.  If the point works then the interval between the zeros is okay.  So domain is $(-4,2)$

\begin{equation} \label{ex:2}
q(x) = 3 + \sqrt[]{x+2}
\end{equation}
Find domain and range of this function.

Note the domain for this function is when $x\geq -2$ or in interval notation $[-2, \infty) $ and the range is $[3,\infty)$.

\begin{marginfigure}
  \includegraphics[width=\linewidth]{2-1Example1.png}
  \caption{In graph you can see the domain and range as where function \ref{ex:2} stops and begins.}
  \label{fig:ex1}
\end{marginfigure}

Find out the value of the function for each of the following value of x
\begin{equation}
g(x) = 3x^2+4x-1
\end{equation}
a) $g(3)$ (ans is 38) b) $g(-a)$ (ans is $3a^2-4a-1$) c) $\frac{g(a+h) - g(a)}{h}$ (ans is $6a+4+3h$)\sidenote[]{although in the original problem $h=0$ seemed to be a bad value of h when we simplified it then it was no longer a problem}

\section{Quiz 2.1}
Do problems 2-3. Remember that each input must have only one output.



















\end{document}