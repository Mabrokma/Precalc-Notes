\documentclass{tufte-handout}

\title{Section 2.6 Algebra of functions}

\author[AW]{Ammon Washburn}

\usepackage{graphicx} % allow embedded images
  \setkeys{Gin}{width=\linewidth,totalheight=\textheight,keepaspectratio}
  \graphicspath{{graphics/}} % set of paths to search for images
\usepackage{amsmath}  % extended mathematics
\usepackage{booktabs} % book-quality tables
\usepackage{units}    % non-stacked fractions and better unit spacing
\usepackage{multicol} % multiple column layout facilities
\usepackage{lipsum}   % filler text
\usepackage[inline]{enumitem}
\usepackage{fancyvrb} % extended verbatim environments
  \fvset{fontsize=\normalsize}% default font size for fancy-verbatim environments

% Standardize command font styles and environments
\newcommand{\doccmd}[1]{\texttt{\textbackslash#1}}% command name -- adds backslash automatically
\newcommand{\docopt}[1]{\ensuremath{\langle}\textrm{\textit{#1}}\ensuremath{\rangle}}% optional command argument
\newcommand{\docarg}[1]{\textrm{\textit{#1}}}% (required) command argument
\newcommand{\docenv}[1]{\textsf{#1}}% environment name
\newcommand{\docpkg}[1]{\texttt{#1}}% package name
\newcommand{\doccls}[1]{\texttt{#1}}% document class name
\newcommand{\docclsopt}[1]{\texttt{#1}}% document class option name
\newenvironment{docspec}{\begin{quote}\noindent}{\end{quote}}% command specification environment

\newtheorem{mydef}{Definition}
\providecommand{\floor}[1]{\left \lfloor #1 \right \rfloor }

\begin{document}
\maketitle

\begin{abstract}
Learn how properties of functions change as you combine functions
\end{abstract}

\section{Algebra of functions and their domains}
Let f(x) and g(x) be given with domains A and B respectively.  Then
\begin{eqnarray}
(f + g)(x) = f(x) + g(x) & Domain: A \cap B \\
(f - g)(x) = f(x) - g(x) & Domain: A \cap B \\
(fg)(x) = f(x)g(x) & Domain: A \cap B \\
\frac{f}{g}(x) = \frac{f(x)}{g(x)} & Domain: A \cap \{x \in B | g(x) \neq 0 \}
\end{eqnarray}

Examples: Let $f(x) = \sqrt[]{2-x}$ and let $g(x) = \frac{x}{\sqrt{x+3}}$.  Domain of f(x) is $(-\infty,2]$ and domain of g(x) is 
$(-3,\infty)$.  Find the following functions and their domains:

\begin{enumerate*}[label=\Alph*]
\item $(f+g)(x)$
\item $(fg)(x)$
\item $\frac{f}{g}(x)$
\end{enumerate*}

\noindent Answer: A) $\sqrt[]{2-x} + \frac{x}{\sqrt[]{x+3}}$ Domain: (-3,2].  B) $\frac{x \sqrt[]{2-x}}{\sqrt{x+3}}$ Domain: (-3,2]  C) $\frac{\sqrt[]{(2-x)(x+3)}}{x}$ Domain: (-3,0) $\cup$ (0,2]

\subsection{Composition of functions}
\begin{mydef}
The composition of f with g (with domains A and B respectively ) or $f \circ g$ is defined as $f \big ( g(x) \big )$ and the domain of this new function is$ \{ x \in B \mid g(x) \in A \}$
\end{mydef}

Easy way of finding it out is finding the domain of g (B) and intersecting with the domain of the simplified function $f \big ( g(x) \big )$.

\smallskip

\noindent Examples: Let $f(x) = \frac{1}{x-5}$ and $g(x) = \frac{5x + 2}{x}$.  Find $f \circ g$ and $g \circ f$ with their domains.

Answers: $f \circ g = \frac{x}{2}$ with Domain: $(-\infty,0) \cup (0,\infty)$.  $g \circ f = 2 x - 5$ with Domain $(-\infty,5) \cup (5,\infty)$

Notice that $f \circ g \neq g \circ f$ and in general they are not equal.  

\section{Quiz}



\end{document}